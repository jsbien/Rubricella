% https://stackoverflow.com/questions/3736292/in-latex-is-there-a-way-to-include-the-filename-of-a-graphic-into-a-caption

% interchar unisugar xecolor xeindex ucharclasses
% font-linuxlibertine?
% texlive-humanities
% fonts-symbola
%\documentclass[12pt]{report}
\documentclass[12pt]{book}
\listfiles
%\usepackage[margin=1in]{geometry}
\usepackage[a4paper,landscape]{geometry}
%\usepackage{titlesec}
\usepackage{polyglossia}
\defaultfontfeatures{Ligatures=TeX}
\setmainlanguage{polish}
%\setotherlanguage[variant=ancient]{greek}
%\setmainfont{Linux Libertine O}
%\setmainfont{Junicode}
%\setmainfont{Symbola}
%\setmainfont[RawFeature={cv58=}]{JuniusX}
%\setmainfont[RawFeature={cv59=1}]{JuniusX}
%\setmainfont[RawFeature={+cv13=1}]{JuniusX}
%\setmainfont{JuniusX}
\setmainfont{JunicodeTwoBeta-Regular}
\newfontfamily{\Symbola}{Symbola_hint.ttf}
\newcommand{\Ssgl}[1]{{\Symbola #1}}
\usepackage{graphicx}
\usepackage{expex}
% ???:
% ERROR: LaTeX Error: Unknown option `backref' for package `enotez'. 0.10!
% \usepackage[backref]{enotez}
% U= Ungler
\lingset{glhangstyle=none}
\defineglwlevels{Rekonstrukcja,Ubytki,U}
 \newcommand{\bg}{\begingl[everygla={\color{blue}},everyglUbytki={\color{green}},everyglRekonstrukcja={\color{black}},%
 everyglU={\color{brown}}]}
% ???:
\let\numerwiersza\excnt


\usepackage{enotez}
\setenotez{backref}
%\let\footnote=\endnote

\usepackage{varioref}

% https://github.com/psb1558/Junicode-New/issues/3
% \setmainfont[RawFeature={cv13=1}]{JuniusX}
% For the record, this syntax will replace the slashed o with the fourth of the alternates for ø in XeLaTeX with Fontspec:

% \usepackage{fontspec}
% \setmainfont{JuniusX}[RawFeature={+cv16=3}]

% or anywhere in the document:

% \addfontfeature{CharacterVariant={16:3}}

% Note that in Fontspec the variants in features like salt and cvNN are indexed from zero instead of from one so use cv16=3 instead of cv16=4.

% %\newfontfamily{\defaultfont}{Symbola}
% \newfontfamily{\defaultfont}{JuniusX}
% % \newfontfamily{\latinfont}{Symbola}
% \newfontfamily{\latinfont}{JuniusX}
% \newfontfamily{\mufi}{Junicode}

% \usepackage[Latin,PrivateUseArea]{ucharclasses}

% nie działa?:
% \setDefaultTransitions{\defaultfont}{}
% \setTransitionsForLatin{\latinfont}{}
% \setTransitionTo{PrivateUseArea}{\mufi}

% \titleformat{\chapter}[display]
% {\normalfont\filcenter}
%  {\LARGE\MakeUppercase{\chaptertitlename} \thechapter}
%  {1pc} 
%  {\vspace{1pc}%
%    \LARGE}
% \titlespacing{\chapter}
% {0pt}{0pt}{10pt}

% \titleformat{\section}[leftmargin]
%  {\normalfont
%   \vspace{0pt}%
%    \bfseries\Large\filleft}
% {\thesection}{.5em}{} 
% \titlespacing{\section}
% {4pc}{1.5ex plus .1ex minus .2ex}{0pt}

% \renewcommand{\thesection}{\arabic{chapter}:\arabic{section}}

\lingset{glhangstyle=none}

\catcode`\&=11
\catcode`\|=11

% see below
% \usepackage{xcolor}
% black, blue, brown, cyan, darkgray, gray, green, lightgray, lime, magenta, olive, orange, pink, purple, red, teal, violet, white, yellow.
\usepackage{fancyhdr}
\renewcommand{\chaptermark}[1]{}
\renewcommand{\sectionmark}[1]{}

%???:
\newcounter{Apageno}[section]
% Sawicki
\newcounter{Spageno}
\newcounter{Slineno}
% Rekonstrukcja
\newcounter{Rpageno}
\newcounter{Rlineno}
% ubytki
\newcounter{upageno}
\newcounter{ulineno}

% Ungler
\newcounter{Upageno}
\newcounter{Ulineno}


%\setlength{\headheight}{14.49998pt}.
\setlength{\headheight}{15pt}


% Haller
\newcommand{\apage}[1]{\setcounter{Apageno}{#1}\colorbox{yellow}{[#1]}}
% Sawicki
\newcommand{\Spage}[2]{\setcounter{Spageno}{#1}\setcounter{Slineno}{#2}\colorbox{yellow}{[#1-#2]}}
\newcommand{\Rpage}[2]{\setcounter{Rpageno}{#1}\setcounter{Rlineno}{#2}\colorbox{yellow}{[#1-#2]}}
\newcommand{\upage}[2]{\setcounter{upageno}{#1}\setcounter{ulineno}{#2}\colorbox{yellow}{[#1-#2]}}
\newcommand{\Upage}[2]{\setcounter{Upageno}{#1}\setcounter{Ulineno}{#2}\colorbox{yellow}{[#1-#2]}}
% placeholder
\newcommand{\xpage}[0]{{}}

\fancypagestyle{Rparallel}{%
\fancyhf{}%
\fancyhead[L]{\color{black}{Ungler region \theUpageno, wiersz \theUlineno}}%
\fancyhead[R]{{\color{blue}{Sawicki  s.~\theSpageno, w. \theSlineno{} (Haller w. \theApageno)}}}%
%\fancyhead[C]{{\color{black}{ubytki s. \theupageno, w. \theulineno}}}%
\fancyfoot[C]{\thepage}%
}

\newcommand{\br}[1]{\index{br:#1}{\color{magenta}#1}}
\newcommand{\bra}[2]{\index{br:#1}{\color{magenta}#2}}

% polski indeksowany
\newcommand{\pli}[1]{\index{pl:#1}{\color{red}#1}}
% polski tekst niewyróżniony kolorem
\newcommand{\plt}[1]{\index{plt:#1}{#1}}

% łacinskie objaśnienia
\newcommand{\lati}[1]{\index{lat:#1}{\color{olive}#1}}



\usepackage{makeidx}\makeindex
%\usepackage{hvindex}% for \Index
% options clash:
% \usepackage[colorlinks]{hyperref}


\usepackage{hyperref}

\usepackage[dvipsnames]{xcolor}


\def\glslink#1#2{#2}
\def\nullum#1{}
\def\Dziech#1{{\let\index\nullum\color{teal}#1}}
\def\Kucharski#1{{\let\index\nullum\color{orange}#1}}
%\let\gls\relax
\newcommand{\gls}[1]{{\color{olive}#1}}

\let\str\relax
% string Kucharski
\newcommand{\strK}[1]{{\index{strK:#1}\colorbox{yellow}{\textit{#1}}}}
% string Kucharski
\newcommand{\StrK}[1]{{\colorbox{GreenYellow}{#1}}}
% string
\newcommand{\str}[1]{{\index{str:#1}\colorbox{yellow}{#1}}}
% string
\newcommand{\Str}[1]{{\colorbox{GreenYellow}{#1}}}
% string
\newcommand{\strA}[1]{{\index{strA:#1}\colorbox{yellow}{#1}}}
% string
\newcommand{\strB}[1]{{\index{strB:#1}\colorbox{yellow}{#1}}}
\let\Index\relax

% https://tex.stackexchange.com/questions/16765/biblatex-author-year-square-brackets
% \usepackage[style=authoryear,backend=biber,backref=true,urldate=short]{biblatex}
\usepackage[style=ext-authoryear,backend=biber,backref=true,urldate=short]{biblatex}
\DeclareOuterCiteDelims{parencite}{\bibopenbracket}{\bibclosebracket}
\addbibresource{PT.bib}
\DefineBibliographyStrings{polish}{%
  byeditor = {red.},
  urlseen = {dostęp},
}

% http://tex.stackexchange.com/questions/166337/quotation-mark-quotation-sign-xelatex-polyglossia-csquotes
% undefined - sprawdzić !!!!!!!!!!!!!!!!!!!!!!!!!!!!!!!!!!!!!!!!!!!!!!!!!!!!!!!!!!!!!!!!!!!!!!!!!!!!!!!!!!!!!!!!!!!!!!!!
% \DeclareQuoteStyle{polish}% I looked it up on Wikipedia, no idea if it's right
%   {\quotedblbase}
%   {\textquotedblright}
%   [0.05em]
%   {\textquoteleft}
%   {\textquoteright}

\catcode`\&=11
\catcode`\|=11
\catcode`\_=11

% nie działa:?
\renewcommand{\topfraction}{0.9}
\renewcommand{\floatpagefraction}{0.9}	% require fuller float pages


\author{Janusz S. Bień (red.)}

\title{{{\color{blue}Rubricelli Hallera z 1511 wg Sawickiego}}\\
  \color{black}{Rubricella Unglera z 1511 r.  --- transkrypcja}\\
    \color{green}{Rubricella Unglera z 1511 r. --- uzupełnienia ubytków (wg Sawickiego)}\\
    \color{brown}{Rubricella Unglera z 1511 r.}}

\date{\today}

\begin{document}

\maketitle{}

\newpage

\pagestyle{Rparallel}
  
% ̅ -> ̄
% OVERLINE nie jest literą i psuje segmentacje na słowa!
% ſƀ͡ gdzie drugie wystąpineie???



\ex \bg
\gla
\Spage{50}{3}\apage{1}
Ut ordo divini officii per clerum sanctae ecclesiae \apage{2} nostrae totiusque 
dyocesis {}
//
\glRekonstrukcja
{}
Ut oꝛdo divini officii per clerum sanctae ecclesiae {} nostrae  totiusque 
dyocesis [Cracoviensis]
//
\glUbytki
\upage{0}{0}
Ut ordo divini {} {} {} {} {} {} {} {} dyocesis [Cracoviensis
//
\glU
\Upage{1}{1}
UT oꝛꝺo ꝺiuini officij per clerũ ſancte eccleſie {} noſtre  totiuſꝙ
dyoceſis {}
//
\endgl
\xe

\ex \bg
\gla
{}
rite et legitime peragatur ad {lau⸗ \apage{3} dem} et reverentiam 
omnipotentis Dei honoremque universalis curiae caelestis, reverendus 
pater dominus Berhnardinus Gallus de
//
\glRekonstrukcja
{}
rite et {��������} {���������} ad laudem et reverentiam 
omnipotentis Dei honoremque universalis curiae caelestis, reverendus 
pater dominus Bernar[dinus] {} {}
//
\glUbytki
{}
rite et [legitime peragatur] ad {} {}
{} {} {} {} {} {} {} {} {} {}
Bernhar[dinus Gallus de] 
//
\glU
\Upage{1}{2}
rite ⁊ {���} {���} aꝺ lauꝺē ⁊ reuerentiā om̄ipotētis ꝺei honoꝛē vniuerſalis curie celeſtis Ren⚠erenꝺus pater ꝺn̄s Bernar
//
\endgl
\xe


\ex \bg
\gla
{}
Jadra, custos vicariusque \apage{4} in spiritualibus* et officialis ecclesiae praedictae cathedralis Cracoviensis 
generalis.
Rubricellam  praesentem ad annum Domini 1511
//
\glRekonstrukcja
{}
Jadra, custos [vi]cari[us]que {} in spiritualibus* et officialis ecclesiae praedictae cathedralis Cracoviensis 
generalis.
Rubricellam  praesentem ad annum Domini Mil[?]
//
\glUbytki
{}
Jadra custos [vi]cari[us]que
{} {} {} {} {} {} {} {} {} {}
{} {} {} {}
domini  1511
//
\glU
\Upage{1}{3}
Jaꝺdra. Cuſto\Ssgl{⧺} \Ssgl{⧺}ca⋯ꝙ {} in ſpũalibꝰ ⁊ officialis eccꝉie p̄⚠ꝺicte Katheꝺꝛalis Cracouien̄ gn̄alis. Rubricellam  pꝛeſentē aꝺ annum ꝺn̄i Mil\Ssgl{⧺}
//
\endgl
\xe


\ex \bg
\gla
{}
{} {}  Tertium 
post bisextilem \apage{5} currentem debite ordinatam ac! institutam mandavit 
subscripto modo teneri et per clerum omnem praedictae dyocesis Cracoviensis
{}
//
\glRekonstrukcja
{}
{?} undecimi  Tertium 
post bisextilem {} currentem debite ordi[natam] ac institutam mandavit 
subscripto modo teneri et per clerum omnem praedictae dyo[cesis]
{}
//
\glUbytki
{}
{} {} {} {} {} {} {}
debite ordin[atam] ac
{} {} {} {} {} {} {} {} {}
predicte[s!] dyo[cesis Cracoviensis]
//
\glU
\Upage{1}{4}
\Ssgl{⧺}flmū vnꝺecimū  Terciū poſt biſextilē {} currētē ꝺebite oꝛdin��� ac inſtitutā. māꝺauit ſƀ͡ſcripto moꝺo teneri. ⁊ ꝑ  clerum om̄eꝫ p̄\Ssgl{⚠}ꝺicte ꝺyo
{}
//
\endgl
\xe



\ex \bg
\gla
{}
observari. Cuius quidem \apage{5}
anni ciclus solaris 8, littera dominicalis 
E, aureus numerus 11, intervallum novem {} ebdomadarum, concurrentes 4, 
{I\ldots} {} {}
//
\glRekonstrukcja
{}
observari. Cuius quidem {}
anni ciclus solaris 8, littera dominicalis 
E, aureus numerus 11, intervallum! {} {9?!} ebdomadarum, concurrentes 4, 
{I\ldots} {} {}
//
\glUbytki
{}
observari
{} {} {} {} {} {} {} {} {}
{} {} {} {} {} {} {} {}
concurrentes 4. I[ndictio 14 Festa]
//
\glU
\Upage{1}{5}
obſeruarı. Cuius quiꝺē {} anni. Ciclus ſolaris 8. Littera ꝺn̄icalis E. Aureus nn\Ssgl{⚠}merꝰ. 11, Interualiū {} {9\Ssgl{⚠}}. Ebdomadarum Cōcurrentes. 4. I\Ssgl{⧺} {} {}
//
\endgl
\xe


\ex \bg
\gla
{}
denique mobilia in his syllabis celebrabuntur. Septuagesima in illa sillaba. {Iu li con}.
Sexagesima. In illa. {Trum ma thi}. Quinquagesima in illa {ci us o}
{} {} {} {}
//
\glRekonstrukcja
{}
denique mobilia in his syllabis celebrabuntur. Septuagesima in illa {}.  Iulicon.
Sexagesima. In illa. Trummathi. Quinquagesima in illa {ci us o}
{} {} {} {}
//
\glUbytki
{}
denique
{} {} {} {} {} {} {} {} {}
{} {} {} {} {}
Quinquagesima in illa: [ci
us o,
//
\glU
\Upage{1}{6}
deni mobilia in his ſillabis celebꝛabūtur Septuagema in illa {}.  Iulicon. Sexagema. In illa. Trūmmathı. Qui\Ssgl{⚠}quagema in illa
//
\endgl
\xe



\ex \bg
\gla
{}
Quadragesima in illa: co ra tur. Pascha: {san cti} 
que. Rogationes: ur ban ni. Penthecosten: iun pri mi. Corporis Christi pro dos al. Adventus Domini: andr
{de cem}//
\glRekonstrukcja
{}
[Quad]ragesima in illa: co ra tur. Pascha: {sancti} 
que. Rogationes: ur ban ni. Penthecosten: iun pri mi. Coꝛporis Christi pro dos al. Adventus Domini: andr
{} {}
//
\glUbytki
{}
[Quad]ragesima.
{} {} {} {} {}
Pascha: {[s]an[cti} que Roga[tiones]
{} {} {} {} {} {} {} {} {}
{} {} {} {} {}
Andr [decem]
//
\glU
\Upage{1}{7}
\Ssgl{⧺}rageima. In illa: Co ra tur. Paſcha: ��� . \Ssgl{⧺}tiōes. Ur ban in\Ssgl{⚠}. Penthecoſten. Iun pri mi. Corp̄\Ssgl{⚠}is xꝑi pro t\Ssgl{⚠}os al. Aduentus ꝺn̄. Anꝺr
//
\endgl
\xe



\ex \bg
\gla
{}
Item
vigilia Epiphanaie veniens
in diem dominicam ibidem 
teneatur
//
\glRekonstrukcja
{}
[I]tem
vigi[lia Epiphan]ie veniens
in diem dominicam ibidem 
teneatur
//
\glUbytki
{}
[It]em
vigi[lia Epiphan]ie veniens
{} {} {} {}
teneatur
//
\glU
\Upage{2}{1}
\Ssgl{⧺}tem vigi\Ssgl{⧺} \Ssgl{⧺}ie veniēs in ꝺiē ꝺn̄icū\Ssgl{⚠} ıbiꝺē teneatur
//
\endgl
\xe



\ex \bg
\gla
{}
ita quod sabbato ad vesperas antiphonae Tecum principium
//
\glRekonstrukcja
{}
[it]a quod sabbato [ad] [vesperas] [antiphonae] Tecum principium
//
\glUbytki
{}
[it]o! quod sabbato [ad vesperas [an]tiphne. Tecum principium
//
\glU
\Upage{2}{2}
\Ssgl{⧺}a \Ssgl{⚠} ſabbato ⋯ ⋯ \Ssgl{⧺}phne\Ssgl{⚠}. Tecū pꝛincipium
//
\endgl
\xe



\ex \bg
\gla
{}
cum aliis antiphonis et psalmis et ad finem prout rubrum viatici docet. 
//
\glRekonstrukcja
{}
[cum?] aliis antiphonis et psalmis et ad finem prout rubrum viatici do⸗ 
//
\glUbytki
{}
[cum] aliis
{} {} {} {} {} {} {} {} 
vratici do[ce]t
//
\glU
\Upage{2}{3}
\Ssgl{⧺}i alijs an̄ijs ⁊ pſalmis ⁊ aꝺ finē ꝓut rubꝛo vratici ꝺo⸗ 
//
\endgl
\xe



\ex \bg
\gla
{}
[do]cet. In matutinis invitatorium, hymni, antiphonae, psalmi, responsoria
//
\glRekonstrukcja
{}
[ce]t. In matutinis. Invitatorium: hymni. antiphonae. psalmi. resp[on]s[oria] 
//
\glUbytki
{}
[ce]t. In matutinis
{} {} {} {}
resp[on]s[oria]
//
\glU
\Upage{2}{4}
\Ssgl{⧺}t. In matutīs. Inuita: hym̄i. an̄e. pſalmi. reſp\Ssgl{⧺}
//
\endgl
\xe



\ex \bg
\gla
{}
prut habentur in die Circumcisionis, cum sex lectionibus de sermone
//
\glRekonstrukcja
{}
[pr]ut habentur in die Circumcisionis, cum sex lectionibus de sermone
//
\glUbytki
{}
[pr]ut
//
\glU
\Upage{2}{5}
\Ssgl{⧺}ut habētur in ꝺie Circūciōis, cū sex lcōibꝰ  ſermone
//
\endgl
\xe

\ex \bg
\gla
{}
aliquo et
omelia Defuncto Herode, responsoria Ecce
//
\glRekonstrukcja
{}
[ali]quo et
omelia defuncto herode, Responsoria Ec⸗
//
\glUbytki
{}
???
//
\glU
\Upage{2}{6}
\Ssgl{⧺}quo ⁊ omelia. Defuncto heroꝺe. Reſpōſoꝛia. Ec⸗
//
\endgl
\xe




\ex \bg
\gla
{}
Ecce Agnus Dei, omnia ut in dominica praeterita {} cum laudibus, prout ibidem
//
\glRekonstrukcja
{}
[Ecce] Agnus Dei, omnia ut in {} praeterita dominica cum laudibus, [prout ibidem]
//
\glUbytki
{}
//
\glU
\Upage{2}{7}
⋯ agnus ꝺei. om̄ia vt in ⋯ p̄\Ssgl{⚠}terita ꝺn̄ica cu\Ssgl{⚠} ⋯ ⋯ ⋯
//
\endgl
\xe



\ex \bg
\gla
{}
ibidem est in vigilia terminatum.
{} Item dominica infra octavas
//
\glRekonstrukcja
{}
ibidem est in vigilia terminatum.
� Item dominica infra octavas
//
\glUbytki
{?}
//
\glU
\Upage{2}{8}
\Ssgl{⧺}ē eſt in vigilia terminatū. ⁌ Itē ꝺn̄ica iu\Ssgl{⚠}fra octās.
//
\endgl
\xe



\ex \bg
\gla
{}
 Epiphaniarum[???] currens in {} loco suo
teneatur et totum de Epiphania fiet
//
\glRekonstrukcja
{}
 [Epiphaniarum???] currens in suo loco {}
teneatur et totum de Epiphania fiet
//
\glUbytki
{?}
//
\glU
\Upage{2}{9}
\Ssgl{⧺}ph̵ie currens in ſuo loco tena\Ssgl{⚠}tur ⁊ totū ꝺe eph̵ia fiet
//
\endgl
\xe



\ex \bg
\gla
{}
praeter omeliam, capitulum, antiphonam ad Benedictus et orationem,
//
\glRekonstrukcja
{}
[p]raeter omeliam. capitulum, antiphonam ad Benedictus et orationem,
//
\glU
\Upage{2}{10}
\Ssgl{⧺}ꝛeter omeliā. capꝉƷ\Ssgl{⚠}. antiphonā aꝺ bn̄ꝺictus. ⁊ or̄oem
//
\endgl
\xe



\ex \bg
\gla
{}
quae de dominica dicentur, comme.[-?]morationem de Epiphania et de
//
\glRekonstrukcja
{}
[qua]e de dominica dicentur, commemorationem de Epiphania et de
//
\glU
\Upage{2}{11}
\Ssgl{⧺}ue ꝺe ꝺn̄ica ꝺicētur. commemoꝛationē ꝺe eph̵ia ⁊ ꝺe
//
\endgl
\xe




\ex \bg
\gla
{}
dominica[???] faciendo, in secundis autem
vesperis omnia de octava fiant 
//
\glRekonstrukcja
{}
[dominica][???] faciendo, in secundis autem
[vesperis omnia de octava fiant]
//
\glU
\Upage{2}{12}
\Ssgl{⧺}n̄a facienꝺo in ſcis aute\Ssgl{⧺} ⋯ ⋯ ⋯ ⋯
//
\endgl
\xe



\ex \bg
\gla
{}
commemorationes dominicae et de dominica[???] apponendo.
{} Item Purificationis
//
\glRekonstrukcja
{}
[com]memorationes dominicae et de dominica[???] apponendo.
� Item Purifi
//
\glU
\Upage{2}{13}
\Ssgl{⧺}memoratiōes ꝺn̄ice et{} ꝺm̄\Ssgl{⚠}e apponēꝺo. ⁌ Itē Purifi⋯
//
\endgl
\xe



\ex \bg
\gla
{}
[[Purifi]][c]ationis sanctae Mariae in dominicam[dominicum] veniens ibidem
teneatur
//
\glRekonstrukcja
{}
[[Purifi]][c]ationis sanctae Mariae in dominicam[dominicum] veniens ibidem
teneatur
//
\glU
\Upage{2}{14}
\Ssgl{⧺}atiōis ſctē Marie in ꝺn̄icū veniens ibiꝺem teneatur.
//
\endgl
\xe



\ex \bg
\gla
{}
et omelia naviculae ad feriam sextam differatur.
{} Item
//
\glRekonstrukcja
{}
et omelia naviculae ad feriam sextam differatur.
{�} Item
//
\glU
\Upage{2}{15}
Et omelia Nauicule. aꝺ feriā ſextam differatur. ⁌ Itē
//
\endgl
\xe



\ex \bg
 \gla
{}
Appoloniae martyris in dominicam  veniens ibidem teneatur,
omelia
//
\glRekonstrukcja
{}
[A]ppoloniae martyris in dominicam  veniens ibidem teneatur,
omelia
//
\glU
\Upage{2}{16}
\Ssgl{⧺}ppolonie mr̄is in ꝺn̄icā  veniēs ibiꝺē teneatur omelia
//
\endgl
\xe



\ex \bg
\gla
{}
vero seminis ultima feria tertia proxima suppleatur.
{} Item vigilia
//
\glRekonstrukcja
{}
[ve]ro seminis ultima feria tertia proxima suppleatur.
� Item vi
//
\glU
\Upage{2}{17}
\Ssgl{⧺}ꝛo ſeminis vltīa\Ssgl{⚠} feria tercia ꝓxī\Ssgl{⚠}a ſuppleatur. ⁌ Itē Ui
//
\endgl
\xe



\ex \bg
\gla
{}
vigilia sancti Mathiae in dominicam Sexagesimae alias Exurge
veniens
//
\glRekonstrukcja
{}
gilia sancti Mathiae in dominicam Sexagesimae alias Exurge
ve
//
\glU
\Upage{2}{18}
\Ssgl{⧺}gilia ſancti Mathie ī\Ssgl{⚠} ꝺn̄icā Sexageme aꝉs\Ssgl{⚠} Exurge ve\Ssgl{⧺}
//
\endgl
\xe



\ex \bg
\gla
{}
veniens ibidem omelia eiusdem pro nona lectione imponatur.
//
\glRekonstrukcja
{}
[n]iens ibidem omelia eiusdem pro nona lectione imponatur.
//
\glU
\Upage{2}{19}
\Ssgl{⧺}iens. ibiꝺē omelia ei\Ssgl{⚠}usꝺem ꝓ nona lcōe imponatur
//
\endgl
\xe



\ex \bg
\gla
{} {} Item Translatio sancti. Venceslai feria tertia Carnisprivii
//
\glRekonstrukcja
{}
� Item Translatio sancti. Venceslai feria tertia Carnisprivii???
//
\glU
\Upage{2}{20}
⁌ Itē Tranſlatio ſancti wēceſlai feria tercia Carniſp̄/
//
\endgl
\xe



\ex \bg
\gla
{}
Carnisprivii veniens
ibidem teneatur, ubi matura missa sollemniter de
//
\glRekonstrukcja
{}
[Carnisprivii]ui? veniens
ibidem teneatur, ubi matura missa sollemniter de
//
\glU
\Upage{2}{21}
\Ssgl{⧺}ui veniē\Ssgl{⚠}s ibiꝺē teneatur vbi matura miffa ſolen\Ssgl{⚠}niter ꝺe
//
\endgl
\xe



\ex \bg
\gla
{}
sancto Stanislao patrono nostro glorioso decantetur.
//
\glRekonstrukcja
{}
sancto Stanislao patrono nostro glorioso decantetur.
//
\glU
\Upage{2}{22}
ſancto Staniſlao patron\Ssgl{⚠}o noſtro gloꝛioſo deca\Ssgl{⚠}ntetur
//
\endgl
\xe



\ex \bg
\gla
{}
{} Item Ciruli et Methudii veniens in dominicam Invocavit
//
\glRekonstrukcja
{}
� Item Ciruli et Methudii veniens in dominicam Invocavit
//
\glU
\Upage{2}{23}
⁌ Itē Cirullij\Ssgl{⚠} ⁊ Methudij veniens in ꝺn̄icā Inuoca\Ssgl{⚠}uit
//
\endgl
\xe



\ex \bg
\gla
{}
ad feriam
secundam differatur.
{} Item Vincentii in[???] dominicam Judica
//
\glRekonstrukcja
{}
ad feriam
secundam differatur.
� Item Vincentii in[???] dominicam Ju
//
\glU
\Upage{2}{24}
aꝺ feriā ſcam differatur. ⁌ Itē Uincenty\Ssgl{⚠} ı\Ssgl{⚠} ꝺnicā\Ssgl{⚠} Ju\Ssgl{⧺}
//
\endgl
\xe


\ex \bg
\gla
{}
Judica veniens ad feriam secundam
in crastinum differatur.
//
\glRekonstrukcja
{}
[[Ju]]dica veniens ad feriam secundam
in crastinum differatur.
//
\glU
\Upage{2}{25}
\Ssgl{⧺}ꝺica veniens aꝺ feriā ſcam in craſtinum ꝺifferatur.
//
\endgl
\xe



\ex \bg
\gla
{}
{} Item Thiburtii et Valeriani feria secunda post Palmas veniens as[???]
sabbatum proximum anteponatur.
//
\glRekonstrukcja
{}
� Item Thiburtii et Valeriani feria secunda post Palmas veniens ad[???]
sabbatum proximum ante
//
\glU
\Upage{3}{1}
⁌ Itē Tiburtij\Ssgl{⚠} et Ualeriani feria ſca poſt palmas veniens aꝺ ſabbatū ꝓximū ante\Ssgl{⧺}
//
\endgl
\xe



\ex \bg
\gla
{}
anteponatur.
{} Item sancti Adalberti feria quarta infra octavas Resurrectionis
veniens
//
\glRekonstrukcja
{}
[[ante]]ponatur.
� Item sancti Adalberti feria quarta infra octavas Resurrectionis
veniens
//
\glU
\Upage{3}{2}
\Ssgl{⧺}ponatur. ⁌ Itē Sancti aꝺalberti feria quarta infra octauas reſurrectiōis veniēs
//
\endgl
\xe



\ex \bg
\gla
{}
ad dominicam Conductus Paschae differatur et omelia dominicalis {} feria secunda pro tertia lectione
//
\glRekonstrukcja
{}
ad dominicam Conductus Paschae differatur et omelia dominicalis {}feria secunda pro tertia lectione
//
\glU
\Upage{3}{3}
a\Ssgl{⚠}ꝺ ꝺn̄icā ↄꝺuctus paſche differatur. ⁊ omelia ꝺn̄icalı\Ssgl{⚠}s in feriā ſcam ꝓ tertia lcōne
//
\endgl
\xe



\ex \bg
\gla
{}
tenenda postponatur, sancto Vitali
in loco permanente.
{} Item Georgii feria quinta infra
//
\glRekonstrukcja
{}
tenenda postponatur, sancto Vitali
in loco permanente.
� Item Georgii feria quinta infra
//
\glU
\Upage{3}{4}
tenēꝺa poſtpou\Ssgl{⚠}atur. Sctō Uitali in loco ꝑmanēte ⁌ Itē georgij feria quinta infra
//
\endgl
\xe




\ex \bg
\gla
{}
octavas Paschae veniens ad feriam
quartam post festum sancti Johannis{???} ante Portam Latinam 
//
\glRekonstrukcja
{}
octavas Paschae veniens ad feriam
quartam post festum sancti Johannis{???} ante Portam Latinam 
//
\glU
\Upage{3}{5}
octauas paſche veniens aꝺ feriā quartā poſt feſt\Ssgl{⚠}ū ſancti Jo\Ssgl{⚠}annis ante poꝛtā latinā 
//
\endgl
\xe





\ex \bg
\gla
{}
transponatur.
{} Item Marci evangelistae feria sexta infra octavas !!!Resurrectionis!!! {}
veniens ad
//
\glRekonstrukcja
{}
transponatur.
� Item Marci evangelistae feria sexta infra octavas {} pasche???
veniens ad
//
\glU
\Upage{3}{6}
tranſponatur. ⁌ Itē Marci Euāngeliſte feria sexta infra octaus\Ssgl{⚠} {} paſche veniēs aꝺ
//
\endgl
\xe



\ex \bg
\gla
{}
feriam quartam post Conductum Paschae cum toto officio,
ieiunio et processione differatur.
//
\glRekonstrukcja
{}
feriam quartam post Conductum Paschae cum toto officio,
ieiunio et processione differatur.
//
\glU
\Upage{3}{7}
feriā quartā poſt ↄꝺuctū paſche cū toto officio Ieiunio ⁊ ꝓceffione\Ssgl{⚠} ꝺifferatur.
//
\endgl
\xe



\ex \bg
\gla
{}
{} Item Floriani veniens in dominicam[dominicum] Misericordia ibidem teneatur
et omelia dominicae in crastino
//
\glRekonstrukcja
{}
� Item Floriani veniens in dominicam[dominicum] Misericordia ibidem teneatur
et omelia dominicae in cra�
//
\glU
\Upage{3}{8}
⁌ {} floꝛiani veniēs in ꝺn̄icā Miſerico\Ssgl{⚠}ꝛdia ibidem teneatur. Et omelia ꝺm̄ice in cra⸗
//
\endgl
\xe




\ex \bg
\gla
{}
crastino expleatur.
{} Item octava sancti Floriani in aliam dominicam veniens videlicet
Jubilate
//
\glRekonstrukcja
{}
crastino expleatur.
� Item octava sancti Floriani in aliam dominicam veniens videlicet
Jubilate
//
\glU
\Upage{3}{9}
\Ssgl{⧺}ſtino expleatur. ⁌ Ite\Ssgl{⚠} octā ſancti floꝛiani in aliā ꝺn̄icam veniens videlicet J\Ssgl{⚠}ubilate
//
\endgl
\xe


\ex \bg
\gla
{}
ibidem teneatur et omelia dominicalis ad feriam sextam differatur.
{} Item Zophiaead ad!!! {} octavam 
//
\glRekonstrukcja
{}
ibidem teneatur et omelia dominicalis ad feriam sextam differatur.
� Item Zophiae {} in octavam 
//

\Upage{3}{10}
ibidē teneatur. ⁊ omelia ꝺn̄icalis aꝺ feriā sextam differatur. ⁌ Itē Zophie {} in octā\Ssgl{⧺}
//
\endgl
\xe



\ex \bg
\gla
{}
octavam sancti Stanis[lai veniens!!! {} ibidem][dziura] teneatur et octava Sancti Stanislai [feria quarta praeterita][dziura]
//
\glRekonstrukcja
{}
{} sancti Stanis[lai {} cadens??? [ibidem][dziura] teneatur et octava Sancti Stanislai feria quarta praeterita???
//
\glU
\Upage{3}{11}
{} ſancti Stanisſlai {} cadēs??? ⋯ ⋯ Et octā ſancti Staniſlai feria quarta �\Ssgl{⚠}\Ssgl{⚠}terita
//
\endgl
\xe



\ex \bg
\gla
{}
peragatur
et tamen [omel]ia[dziura] eiusdem octavae ipso die Zophiae in matutinis et
vesperis teneatur.
//
\glRekonstrukcja
{}
peragatur
et tamen [[omel]ia[dziura]???] eiusdem octavae ipso die Zophiae in matutinis et
vesperis teneatur.
//
\glU
\Upage{3}{12}
ꝑagatur. Et tn̄ memoꝛ\Ssgl{⧺} ⋯ ⋯ ⋯ ⋯ Zophie in matutīs ⁊ vesꝑis teneatur.
//
\endgl
\xe


\ex \bg
\gla
{}
{} Item. Urbani in dominicam Rogationum veniens ibidem teneatur
et omelia dominicalis in crastinum
//
\glRekonstrukcja
{}
⁌ Item. Urbani in dominicam Rogationum veniens ibidem teneatur
et omelia dominicalis in cra⸗
//
\glU
\Upage{3}{13}
⁌ Itē Urbani in ꝺn̄icā rogationū veniēs ibiꝺē teneatur. ⁊ omelia ꝺn̄icalis in cra⸗
//
\endgl
\xe



\ex \bg
\gla
{}
crastinum differatur et ibidem omelia diei pro
ultima lectione dicatur.
{} Item dominica infra octavas
//
\glRekonstrukcja
{}
ſtinn̄\Ssgl{⚠} differatur et ibidem omelia diei pro
ultima lectione dicatur.
⁌ Item dominica infra octavas
//
\glU
\Upage{3}{14}
ſtinn̄\Ssgl{⚠} ꝺifferatur. ⁊ ibiꝺē omelia ꝺiei ꝓ vltima lcōe ꝺicatur. ⁌ Itē ꝺn̄ica infra octās
//
\endgl
\xe



\ex \bg
\gla
{}
Ascensionis currens in loco suo teneatur.
{} Item Erasmi in feriam tertiam infra octavas Ascensionis
//
\glRekonstrukcja
{}
ascensionis currens in loco suo teneatur.
� Item Erasmi in feriam tertiam infra octavas Ascen
//
\glU
\Upage{3}{15}
{}
a\Ssgl{⚠}ſcenſiōis currēs in loco suo teneatur. ⁌ Itē Eraſmi\Ssgl{⚠} in feriā terciā infra octās aſcē
//
\endgl
\xe



\ex \bg
\gla
{}
Ascensionis veniens
ibidem teneatur. 
{} Item Barnabae apostoli ad feriam quartam infra octavas
//
\glRekonstrukcja
{}
[[Ascen]]sionis veniens
ibidem teneatur. � Item Barnabae apostoli ad feriam quartam infra octavas
//
\glU
\Upage{3}{16}
{}
\Ssgl{⧺}ꝺi\Ssgl{⚠}onis veniēs ibiꝺē teneatur. ⁌ Itē Barnabe apꝉ\Ssgl{⚠}i aꝺ feriam quartā infra octās
//
\endgl
\xe



\ex \bg
\gla
{}
Penthecosten veniens ad feriam sextam post octavas Ascensionis anticipetur.
Festa vero trium
//
\glRekonstrukcja
{}
penthecosten veniens ad feriam sextam post octavas ascensionis anticipetur.
festa vero trium
//
\glU
\Upage{3}{17}
penth\Ssgl{⚠}costen veniēs ad fe\Ssgl{⧺} ⋯  poſt octās aſcenōi\Ssgl{⚠}s anticipetur feſta vero triū
//
\endgl
\xe




\ex \bg
\gla
{}
lectionum infra octavas Penthecoste cadentia
silentio praetereant, solemnitate Penthecosten
//
\glRekonstrukcja
{}
lectionum infra octavas penthecoste cadentia
silentio??? praetereant, solemnitate penthecosten
//
\glU
\Upage{3}{18}
lcōnū infra octās pentheco\Ssgl{⧺} cadentia\Ssgl{⚠} silentio\Ssgl{⚠} pꝛetereant ſolem\Ssgl{⚠}nitate penthecoſten
//
\endgl
\xe



\ex \bg
\gla
{}
id exigente.
{} Item Viti et Modesti in diem Sanctissimae Trinitatis veniens in
crastinum
//
\glRekonstrukcja
{}
id exigente.
� Item Viti et modesti in diem sanctissimae trinitatis veniens in
crastinum
//
\glU
\Upage{3}{19}
iꝺ exigente. ⁌ Itē V\Ssgl{⚠}iti ⁊ moꝺeſti in diē ſanctimae trinitatis veniēs in craſtinū
//
\endgl
\xe


\ex \bg
\gla
{}
differatur, ita quod cantatis secundis vesperis de sancta Trinitate post orationem non conclusam
//
\glRekonstrukcja
{}
differatur. ita quod cantatis secundis vesperis de sancta trinitate post orationem non conclu⸗
//
\glU
\Upage{3}{20}
differatur. ita ꝙ cantatis ſcis vesperis de ſancta trin\Ssgl{⚠}itate poſt or̄oem non conclu⸗
//
\endgl
\xe



\ex \bg
\gla
{}
conclusam a responsorio Isti sunt sancti vesperae de sanctis incipientur et explebuntur. Completorium
//
\glRekonstrukcja
{}
[[conclu]]sam a responsorio. Isti sunt sancti. vespere de sanctis incipientur et explebuntur. Com
//
\glU
\Upage{3}{21}
\Ssgl{⧺}ſam a reſponſoꝛio. Iſti sunt ſancti. veſꝑe ꝺe ſanctis incipiētur ⁊ explebuntur. com\Ssgl{⚠}\Ssgl{⧺}
//
\endgl
\xe



\ex \bg
\gla
{}
Completorium de sancta 
Trinitate dicatur.
Item Gervasii et Protasii veniens in diem Corporis
//
\glRekonstrukcja
{}
Completorium de sancta trinitate dicatur. � Item gervasii et protasii veniens in diem corporis//
\glU
\Upage{3}{22}
\Ssgl{⧺}pletoꝛiū ꝺe sancta trinitate dicatur. ⁌ Itē geruaſij ⁊ ꝓthaſij\Ssgl{⚠} veniens in diē coꝛꝑis
//
\endgl
\xe



\ex \bg
\gla
{}
Christi ad 
feriam tertiam post {} {} Trinitatis anticipetur.
{} Item decem millium militum
//
\glRekonstrukcja
{}
christi ad 
feriam tertiam post festum sancte trinitatis anticipetur.!!!
� Item decem millium mi
//
\glU
\Upage{3}{23}
x\Ssgl{⚠}p̄i aꝺ feriā tertiā poſt feſtū ſancte trinitatis anteponatur. ⁌ Itē Decē miliū mi
//
\endgl
\xe




\ex \bg
\gla
{}
militum dominica infra octavas Corporis 
Christi veniens ad feriam quartam post festum sanctae Trinitatis
//
\glRekonstrukcja
{}
[[mi]]litum dominica infra octavas corporis 
christi veniens ad feriam quartam post festum sancte trinitatis
//
\glU
\Upage{3}{24}
\Ssgl{⧺}li\Ssgl{⚠}tū ꝺn̄ica infra octās coꝛꝑis xp̄i veniēs aꝺ feriā quartā poſt feſtum ſancte trinitatis
//
\endgl
\xe


\ex \bg
\gla
{}
anteponatur, omelia vero eiusdem dominicae pro nona lectione in loco imponatur.
{} Item vigilia sancti
//
\glRekonstrukcja
{}
anteponatur. omelia vero eiusdem!!! dominicae pro nona lectione in loco imponatur.
� Item vigilia sancti
//
\glU
\Upage{3}{25}
anteponatur. omelia v\Ssgl{⚠}o eiuſoē ꝺ̄nice ꝓ nona lcōe ī loco īmponatur. ⁌ Itē vigilia ſcī
//
\endgl
\xe



\ex \bg
\gla
{}
Johannis Baptistae feria secunda infra octavas
Corporis Christi veniens totum de Corpore Christi teneatur
//
\glRekonstrukcja
{}
Johannis baptistae feria secunda infra octavas
corporis christi veniens totum de corpore christi teneatur
//
\glU
\Upage{3}{26}
Joānis baptiſtte feria ſcꝺa infra octās coꝛꝑis xp̄i veniēs totū ꝺe coꝛꝑe xp̄i teneatur
//
\endgl
\xe



\ex \bg
\gla
{}
et omelia
vigiliae pro tertia lectione imponatur. Primae vesperae de sancto Johanne,
praemittendo processionem
//
\glRekonstrukcja
{}
et omelia
vigilie pro tertia lectione imponatur. Primae vesperae de sancto Johanne,
praemittendo processio
//
\glU
\Upage{3}{27}
⁊ omelia vigilie p\Ssgl{⧺}\Ssgl{⚠} ⋯ lcōe īponatur pꝛime veſꝑe ꝺe ſancto Joāne p̄mittēdo ꝓoceio
//
\endgl
\xe



\ex \bg
\gla
{}
processionem cum venerabili Sacramento, cantentur commemorationem de Corpore Christi faciendo. Ad completorium
//
\glRekonstrukcja
{}
[[processio]]nem cum venerabili Sacramento, cantentur commemorationem??? de Corpore christi faciendo. Ad completo⸗
//
\glU
\Upage{3}{28}
\Ssgl{⧺}nē cū vener\Ssgl{⧺} \Ssgl{⧺}ēto cātētur. ↄmemoꝛatiōe ꝺe coꝛꝑe xp̄i faciēꝺo. aꝺ ↄpleto⸗
//
\endgl
\xe



\ex \bg
\gla
{}
completorium psalmi
consueti, hymnus O nimis felix, capitulum??? Panem??? cet?? veritatent, versus
Custodi nos, antiphona
//
\glRekonstrukcja
{}
[[completo]]rium psalmi
consueti, hymnus O nimis felix, Capitulum??? Panem??? et?? veritatent, versus
Custodi nos. an⸗
//
\glU
\Upage{3}{29}
\Ssgl{⧺}riū pſal\Ssgl{⚠}mi ↄsu\Ssgl{⚠}\Ssgl{⧺} \Ssgl{⧺}s O nimis felix, Capꝉ\Ssgl{⚠}m Pacē ⁊ veri: v\Ssgl{⚠} Cuſtoꝺi nos. an⸗
//
\endgl
\xe




\ex \bg
\gla
{}
antiphona ad Nunc dimittis, pro eo preces consuetae cum
oratione Deus qui illuminas.{etc.!!!}
//
\glRekonstrukcja
{}
[[an]]tiphona ad nunc {???} Pro eo Preces consuete cum
oratione Deus qui illuminas. {etc.!!!}
//
\glU
\Upage{3}{30}
\Ssgl{⧺}tiphona aꝺ\Ssgl{⚠} nun\Ssgl{⚠}c ⋯ Pꝛo eo Pꝛeces conſuete cum or̄one. Deus qui illuminas. {⁊c̄}.
//
\endgl
\xe




\ex \bg
\gla
{}
Festum tandem sancti Johannis in crastino solenniter cum toto
suo officio celebretur. Secundae vesperae
//
\glRekonstrukcja
{}
Festum tandem sancti Johannis in crastino solenniter cum toto
{} officio celebretur. Secundae vespere
//
\glU
\Upage{3}{31}
feſtūm tāꝺē ſancti Joa\Ssgl{⚠}n̄is in craſtino ſolēniter cū toto {} officio celebꝛetur ſce veſꝑe
//
\endgl
\xe



\ex \bg
\gla
{}
de sancto Johanne teneatur
cum commemoratione et completo(rio) de Corpore Christi.
Item Johannis et
//
\glRekonstrukcja
{}
de sancto Johanne teneatur???
cum commemoratione??? et completorio) de corpore christi.
Item Johannis et
//
\glU
\Upage{3}{32}
ꝺe .ſ. Joāne teneātur cū ↄmemoꝛatione ⁊ ↄpletoꝛio ꝺe coꝛꝑe xp̄i. Itē Joannis ⁊
//
\endgl
\xe



\ex \bg
\gla
{}
Pauli veniens in octavam Corporis Christi ad
sabbatum proximum??? differatur et ibidem omelia de
//
\glRekonstrukcja
{}
pauli veniens in octavam corporis christi ad
sabbatum proximum??? differatur. et ibidem omelia de
//
\glU
\Upage{3}{33}
pauli veniēs in octauā ⋯ xp̄i aꝺ ſabbatū ꝓxim\Ssgl{⚠}ū ꝺifferatur. ⁊ ıbiꝺē omelia ꝺe
//
\endgl
\xe



\ex \bg
\gla
{}
vigilia apostolorum
Petri et Pauli pro nona lectione imponatur.
{} Item Ladislai regis feria sexta
//
\glRekonstrukcja
{}
vigilia apostolorum
petri et pauli pro nona lectione imponatur.
� Item Ladislai regis feria se/
//
\glU
\Upage{3}{34}
vigilia a\Ssgl{⚠}pꝉoꝝ petri ⁊ pauli ꝓ nona lcōe imponatur. ⁌ Itē Ladiſlai regis feria ſe/
//
\endgl
\xe



\ex \bg
\gla
{}
sexta in crastino octavae Corporis Christi
veniens ibidem teneatur, ita quod cantatis de Corpore Christi
//
\glRekonstrukcja
{}
sexta in crastino octavae Corporis Christi
veniens ibidem teneatur, ita quod cantatis de Corpore Christi
//
\glU
\Upage{3}{35}
\Ssgl{⧺}xta in craſtstino oct\Ssgl{⧺} \Ssgl{⧺}ꝛꝑis xp̄i veniens ibiꝺē teneatur ita ꝙ cantatis ꝺe coꝛꝑe xp̄i
//
\endgl
\xe




\ex \bg
\gla
{}
vesperis
et oratione dicta et non conclusa incipiatur responsorium Justum deduxit et sic vesperae
//
\glRekonstrukcja
{}
vesꝑis ⁊ or̄oe ꝺicta ⁊ nō concluſa incipiatur reſpōſoꝛiū. Justū ꝺeꝺuxit ⁊ ſic veſꝑe
//
\glU
\Upage{3}{36}
//
\endgl
\xe



\ex \bg
\gla
{}
ad finem continuentur cum completorio de Corpore Christi.
{} Item Petri et Pauli apostolorum in dominicam veniens
//
\glRekonstrukcja
{}
ad finem continuentur cum completorio de Corpore Christi.
� Item Petri et Pauli apostolorum in dominicam ve
//
\glU
\Upage{3}{37}
⋯ ⋯ \Ssgl{⧺}tinuentur cū ↄp⋯ꝛio  coꝛꝑe xp̄i ⁌ Itē Petri ⁊ Pauli apꝉoꝝ ī ꝺn̄icā ve\Ssgl{⧺}
//
\endgl
\xe



\ex \bg
\gla
{}
veniens ibidem
teneatur et omelia dominicalis pro nona lectione imponatur.
{} Item octava Petri et Pauli {} {}
//
\glRekonstrukcja
{}
[[ve]]niens ibidem
teneatur et omelia dominicalis pro nona lectione imponatur.
� Item octava {} {} {} sanctorum apostolorum???
//
\glU
\Upage{3}{38}
\Ssgl{⧺}n\Ssgl{⚠}ēs ibiꝺē teneatur. ⁊ omelia ꝺn̄icalis ꝓ nona lcōe īponatur ⁌ Itē octā\Ssgl{⚠} {} {} {} ſctōꝝ apꝉoꝝ
//
\endgl
\xe



\ex \bg
\gla
{}
veniens!!! in dominicam {} ibidem teneatur,
omeliam dominicalem ad feriam sextam differendo.
{} Item Dedicatio
//
\glRekonstrukcja
{}
{} in dominicam  veniens ibidem teneatur,
omeliam dominicalem ad feriam sextam differendo.
� Item Dedicatio
//
\glU
\Upage{3}{39}
{} in ꝺn̄icā  veniēs ibiꝺē teneatur omeliā ꝺo: aꝺ feriā sextā differēꝺo ⁌ Itē Deꝺicatio
//
\endgl
\xe



\ex \bg
\gla
{}
nostrae cathedralis ecclesiae certis festis et eorum
octavis praepedita non potuit ri[teac][dziura]!!! legitime
//
\glRekonstrukcja
{}
nostrae cathedralis ecclesie certis festis et eorum
octavis praepedita??? non potuit rite {} legitime
//
\glU
\Upage{3}{40}
\Ssgl{⧺}r̄e \Ssgl{⧺}hedꝛalis eccc\Ssgl{⚠}ꝉie certis feſtis ⁊ eoꝝ octauis p̄peꝺita nō potuit rite ⁊ legitt\Ssgl{⚠}īe
//
\endgl
\xe



\ex \bg
\gla
{}
iuxta antiquam consuetudinem prima dominica post octavas Corporis Christi celebrari, quare
idem
//
\glRekonstrukcja
{}
iuxta antiquam consuetudinem prima dominica post octavas corporis christi celebrari quare
idem
//
\glU
\Upage{3}{41}
iuxta antiquā ↄſuetuꝺinē pꝛīa ꝺn̄ica poſt octauas coꝛꝑis xp̄i celebꝛari quare iꝺē
//
\endgl
\xe



\ex \bg
\gla
{}
festum ad dominicam proximam post octavas Visitationis transponitur celebrandum, in quam etiam
//
\glRekonstrukcja
{}
festum ad dominicam proximam post octavas visitationis transponitur!!! celebrandum, in quam etiam
//
\glU
\Upage{3}{42}
\Ssgl{⧺}ſtū aꝺ ꝺn̄icā ꝓximā poſt octan\Ssgl{⚠}as vitatiōis tranſponitur celebꝛanꝺū\Ssgl{⚠} in quā etiā
//
\endgl
\xe



\ex \bg
\gla
{}
festum sanctae Margarethae cadit,
et nihilominus ibidem Dedicatio celebretur, cui festum
//
\glRekonstrukcja
{}
festum sanctae Margarethe cadit
et nihilominus ibidem Dedicatio celebretur. Cui festum
//
\glU
\Upage{3}{43}
\Ssgl{⧺}tū ſancte Margarethe caꝺit ⁊ nihilominus ibiꝺē ꝺeꝺicatio celebꝛetur. Cui\Ssgl{⚠} feſtū
//
\endgl
\xe



\ex \bg
\gla
{}
sanctae Margarethae cum suo tamen officio tenendum per clerum in crastinum cedat,!!!
festo tamen sanctae
%octavis
//
\glRekonstrukcja
{}
sanctae Margarethae cum suo tamen officio tenendum!!! per clerum in crastinum cedat,!!!
festo tamen sanctae
//
\glU
\Upage{3}{44}
\Ssgl{⧺}ncte Margarethe cū ſuo tm̄ officio tenenꝺo ꝑ clerū in craſtinū ceꝺit feſto tn̄ ſctē
//
\endgl
\xe



\ex \bg
\gla
{}
Margarethae in loco proprio per vulgum celebrando
[per]manente[dziura], ita tamen@@@, quod omelia dominicalis
//
\glRekonstrukcja
{}
Margarethae in loco proprio per vulgum celebrando
permanente ita tamen quod omelia??? dominicalis
//
\glU
\Upage{3}{45}
Margarethae in loco ꝓprio ꝑ vulgū celebꝛāꝺo ꝑmanente ita tn̄ ꝙ omelia ꝺn̄icalis
//
\endgl
\xe



\ex \bg
\gla
{}
pro nona lectione
imponatur, ipso die Dedicationis secundis {} vesperis de Dedicatione cantatis et
//
\glRekonstrukcja
{}
pro nona lectione
imponatur ipso die dedicationis. secundis de vesperis {} dedicatione cantatis et
//
\glU
\Upage{3}{46}
⋯ nona lcōe ī\Ssgl{⚠}mponatur ip̄o dì\Ssgl{⚠}e ꝺeꝺicationis. ſcis {} veſꝑis  ꝺe ꝺeꝺıcatiōe cantatis e\Ssgl{⚠}t
//
\endgl
\xe



\ex \bg
\gla
{}
oratione non conclusa a responsorio Quadam die vesperae de
sancta Margaretha inceptae
//
\glRekonstrukcja
{}
oratione non conclusa a responsorio Quadam die vesperae de
sancta Margaretha incepte
//
\glU
\Upage{3}{47}
⋯ ⋯ \Ssgl{⧺}lusa a reſpōſoꝛio. Quaꝺā ꝺie. veſꝑe ꝺe ſancta Margaretha incepte
//
\endgl
\xe



\ex \bg
\gla
{}
ad finem continuentur, completorio de De-
dicatione tento.
In quibus autem ecclesiis Dedicatio non celebratur, Margarethae
in suo loco teneatur cum transpositione dominicalis omeliae ad feriam
secundam.
Item octava Dedicationis in dominicam veniens ibidem teneatur,secundam.
Item octava Dedicationis in dominicam veniens ibidem teneatur,
omelia vero dominicalis pro IX. lectione in loco imponatur. Aliae vero
ecclesiae Dedicationem non habentes dominicam
//
\glRekonstrukcja
{}
% !!!!!!!!!!!!!!!!!!!!!!!!!!!!!!!!!!!!!!!!!!!!!!!
% ad finem continuentur, completorio de De-
% dicatione tento.
% In quibus autem ecclesiis Dedicatio non celebratur, Margarethae
% in suo loco teneatur cum transpositione dominicalis omeliae ad feriam
% secundam.
% Item octava Dedicationis in dominicam veniens ibidem teneatur,secundam.
% Item octava Dedicationis in dominicam veniens ibidem teneatur,
% omelia vero dominicalis pro IX. lectione in loco imponatur. Aliae vero
% ecclesiae Dedicationem non habentes dominicam
//
\endgl
\xe



\ex \bg
\gla
{}
in suo loco {} tenebunt.
{} Item Christoferi in dominicam 
//
\glRekonstrukcja
{}
in {} loco suo tenebunt?.
� Item Christoph 
//
\glU
\Upage{4}{1}
in {} loco ſuo teu\Ssgl{⚠}ebūt. ⁌ Itē Chriſtoph 
//
\endgl
\xe



\ex \bg
\gla
{}
veniens ibidem teneatur in crastinum
omeliam dominicae 
//
\glRekonstrukcja
{}
[ve]niens ibidem teneatur in crastinum
omeli
//
\glU
\Upage{4}{2}
\Ssgl{⧺}niens ibiꝺē teneatur. ın craſtinū omeli\Ssgl{⧺}
//
\endgl
\xe



\ex \bg
\gla
{}
transferendo.
{} Item Laurentii in dominicam veniens ibidem teneatur et 
//
\glRekonstrukcja
{}
do.
� Item Laurentii in dominicam veniens
//
\glU
\Upage{4}{3}
\Ssgl{⧺}ꝺo. ⁌ Itē Laurencij in ꝺn̄icā veniēs
//
\endgl
\xe



\ex \bg
\gla
{}
omelia
dominicae ad feriam quintam differatur, ibidem 
//
\glRekonstrukcja
{}
omelia
dominicae ad feriam quintam differatu[r, ibidem]
//
\glU
\Upage{4}{4}
omelia ꝺn̄ice aꝺ feriā quintā ꝺifferatu\Ssgl{⧺}
//
\endgl
\xe



\ex \bg
\gla
{}
vigiliae omeliam!!! Assumptionis pro tertia lectione imponendo. Item dominica 
//
\glRekonstrukcja
{}
vigilie {} assumptionis pro tertia lectione impo[nendo. Item dominica] 
//
\glU
\Upage{4}{5}
vigilie {} aumptiōis ꝓ tercia lcōe impo\Ssgl{⧺} ⋯ ⋯
//
\endgl
\xe



\ex \bg
\gla
{}
dominica infra octavas Assumptionis currens totum de Assumptione 
//
\glRekonstrukcja
{}
[[domini]]ca infra octavas Assumptionis currens totum[ de Assumptione] 
//
\glU
\Upage{4}{6}
\Ssgl{⧺}ca infra octās aumptiōis currēs totū ⋯ ⋯
//
\endgl
\xe


\ex \bg
\gla
{}
{} omeliam dominicae pro nona lectione imponendo.
Item 
//
\glRekonstrukcja
{}
teneatur   omeliam dominicae pro nona lectione i[mponendo.
Item]
//
\glU
\Upage{4}{7}
teneatur   omeliā ꝺn̄ice ꝓ nona lctōe ı\Ssgl{⧺} ⋯
//
\endgl
\xe



\ex \bg
\gla
{}
Bartholomaei apostoli in dominicam ve(niens) ibidem teneatur et 
//
\glRekonstrukcja
{}
Bartholomaei apostoli in dominicam ve(niens) ib[idem teneatur et] 
//
\glU
\Upage{4}{8}
Bartholomaei apꝉi in ꝺn̈\Ssgl{⚠}icā veniēs ib\Ssgl{⧺} ⋯ ⋯
//
\endgl
\xe



\ex \bg
\gla
{}
omelia do(minicalis) ad feriam tertiam {} transp(onatur).!!!
Mathebertae
//
\glRekonstrukcja
{}
omelia do(minicalis)!!! ad feriam tertiam differatur [transp(onatur).!!!
Mathebertae]
//
\glU
\Upage{4}{9}
omelia ꝺn̄ice aꝺ feriā tertiā ꝺifferatur ⋯ ⋯
//
\endgl
\xe



\ex \bg
\gla
{}
Mathebertae in do(minicam) veniens ibidem teneatur  cum impositione
//
\glRekonstrukcja
{}
[[Mathe]]berte in do(minicam) veniens ibidem teneatur  [cum impositione]
//
\glU
\Upage{4}{10}
bē\Ssgl{⚠}rte in ꝺn̄icā veniēs ibiꝺē teneatur  ⋯ ⋯
//
\endgl
\xe



\ex \bg
\gla
{}
omeliae de vigilia Nativitatis Mariae pro nona lectione, omelia
//
\glRekonstrukcja
{}
omelie de\Ssgl{⚠} vigilia nativitatis mariae pro\Ssgl{⚠} n[ona lectione, omelia]
//
\glU
\Upage{4}{11}
omelie  vigilia natiuitatis marie ꝓ n\Ssgl{⧺} ⋯ ⋯
//
\endgl
\xe



\ex \bg
\gla
{}
vero dominicalis in crastino ipso die Nativitatis [[virginis-JSB]] Mariae
//
\glRekonstrukcja
{}
vero dominicalis in crastino ipso die nativ[itatis [[virginis-JSB]] Mariae]
//
\glU
\Upage{4}{12}
vero ꝺn̈icalis in craſtino ip̄o ꝺie natiu\Ssgl{⧺} ⋯ ⋯
//
\endgl
\xe



\ex \bg
\gla
{}
Mariae imponatur pro
nona lectione.
Item Exaltationis sanctae
//
\glRekonstrukcja
{}
Mariae imponatur pro
nona lectione.
Item Exaltationis sanctae
//
\glU
\Upage{4}{13}
\Ssgl{⧺}rie imponatur ꝓ nona lcōe. ⁌ Itē E\Ssgl{⧺} ⋯
//
\endgl
\xe



\ex \bg
\gla
{}
Crucis in dominicam veniens ibidem
teneatur omeliam dominicae ad 
//
\glRekonstrukcja
{}
crucis\Ssgl{⚠} in dominicam veniens ibidem
teneatur [omeliam dominicae ad]
//
\glU
\Upage{4}{14}
crn\Ssgl{⚠}cis in ꝺn̄icā veniēs ibiꝺē teneatur ⋯ ⋯ ⋯ 
//
\endgl
\xe



\ex \bg
\gla
{}
feriam quartam Quatuor Temporum
differendo et ibidem omeliam 
//
\glRekonstrukcja
{}
feriam quartam quatuor temporum
differendo [et ibidem omeliam] 
//
\glU
\Upage{4}{15}
feriā quartā quatuoꝛ tꝑm ꝺifferenꝺo. ⋯ ⋯ ⋯
//
\endgl
\xe



\ex \bg
\gla
{}
diei pro tertia lectione imponendo.!!! {}
{} Sabbato vero Quatuor 
//
\glRekonstrukcja
{}
diei pro\Ssgl{⚠}tertia lectione {} imponendo.!!!
� Sab[bato vero Quatuor]
//
\glU
\Upage{4}{16}
ꝺiei ꝓtercia lcōe {} imponatur. ⁌ Sa\Ssgl{⧺}
//
\endgl
\xe



\ex \bg
\gla
Temporum omelia vigiliae sancti Mathaei
pro tertia lectione imponatur.
//
\glRekonstrukcja
temporum\Ssgl{⚠} omelia vigiliae sancti Mathaei
pro\Ssgl{⚠} [tertia lectione imponatur.]
//
\glU
\Upage{4}{17}
tꝑm omelia vigilie ſancti Mathaei ꝓ ⋯ ⋯ ⋯
//
\endgl
\xe



\ex \bg
\gla
{}
imponatur.
{} Item festum sancti Mathaei in dominicam[dominicum] veniens ibidem
//
\glRekonstrukcja
{}
[[impona]]tur.
� Item festum sancti Mathaei in d[ominicam[dominicum] veniens ibidem]
//
\glU
\Upage{4}{18}
tur. ⁌ Itē feſtum ſancti Mathaei in ꝺn̄\Ssgl{⧺} ⋯ ⋯
//
\endgl
\xe



\ex \bg
\gla
{}
teneatur et omelia dominicae ad feriam tertiam differatur.
//
\glRekonstrukcja
{}
teneatur et omelia dominicae ad feriam te[rtiam differatur.]
//
\glU
\Upage{4}{19}
teneatur ⁊ omelia ꝺn̈ice aꝺ feriā te\Ssgl{⧺} ⋯
//
\endgl
\xe



\ex \bg
\gla
{}
{} Item {} Venceslai in do(minicam) veniens ibidem celebretur
//
\glRekonstrukcja
{}
� Item sancti Venceslai in do(minicam) ve[niens ibidem celebretur]
//
\glU
\Upage{4}{20}
⁌ Itē ſancti wenceſlai in ꝺn̄icā ven\Ssgl{⚠}\Ssgl{⧺}
//
\endgl
\xe



\ex \bg
\gla
{}
celebretur et!!! omelia
dominicae pro IX. {} lectione imposita. Item octava
//
\glRekonstrukcja
{}
[celebr]etur {} omelia
dominicae pro\Ssgl{⚠} {} nona lectione impo[sita. Item octava]
//
\glU
\Upage{4}{21}
\Ssgl{⧺}etur {} omelia ꝺn̄ice ꝓ {} nona lcōe impo\Ssgl{⧺} ⋯ ⋯
//
\endgl
\xe



\ex \bg
\gla
{}
sancti Venceslai in aliam do(minicam) veniens ad feriam sextam proximam
//
\glRekonstrukcja
{}
sancti Venceslai in aliam do(minicam) veniens [ad feriam sextam proximam]
//
\glU
\Upage{4}{22}
ſaucti wenceſlai in aliā ꝺn̄icā veniēs ⋯ ⋯ ⋯ ⋯
//
\endgl
\xe



\ex \bg
\gla
{}
proximam anteponatur et nihilominus memoria de octava
//
\glRekonstrukcja
{}
[[proxi]]\Ssgl{⧺}mam anteponatur. Et nihilominus m[emoria de octava]
//
\glU
\Upage{4}{23}
\Ssgl{⧺}mam {ante ponatur}. Et nihilominus �\Ssgl{⧺} ⋯ ⋯
//
\endgl
\xe



\ex \bg
\gla
{}
ad do(minicam) continuetur.
{} Item Commemoratio animarum veniens in do(minicam) ad feriam
secundam 
//
\glRekonstrukcja
{}
ad do(minicam)!!! continuetur.
� Item Commemora[tio animarum veniens in do(minicam) ad feriam
secundam] 
//
\glU
\Upage{4}{24}
aꝺ ꝺn̄icum ↄtinuetur. ⁌ Itē Cōmemoꝛa\Ssgl{⧺} ⋯ ⋯ ⋯ ⋯ ⋯ ⋯ ⋯
//
\endgl
\xe



\ex \bg
\gla
{}
iuxta informationem viatici transponatur.!!! {}
Item do(minica)
//
\glRekonstrukcja
{}
iuxta informationem viatici {} teneatur
[Item do(minica)]
//
\glU
\Upage{4}{25}
iuxta {in foꝛmationē} viatici {} teneatur. ⋯ ⋯
//
\endgl
\xe


\ex \bg
\gla
{}
infra octavas Omnium Sanctorum totum de Omnibus Sanctis teneatur, omelia dominicae pro nona
//
\glRekonstrukcja
{}
infra octavas omnium sanctorum totum de omnibus sanctis teneatur ome[lia dominicae pro nona]
//
\glU
\Upage{5}{1}
infra octauas omniū ſanctoꝝ totū ꝺe om̄ibus ſanctis tenetur om\Ssgl{⧺} ⋯ ⋯ ⋯
//
\endgl
\xe



\ex \bg
\gla
{}
lectione imponendo.
Festa vero infra octavas venientia iuxta dispositionem Summi
Pontificis
//
\glRekonstrukcja
{}
lectione imponendo.
festa vero infra octavas venientia iuxta dispositio[nem Summi
Pontificis]
//
\glU
\Upage{5}{2}
lcōe imponenꝺo. feſta vero infra octauā veniētia iuxta ꝺiſpotō\Ssgl{⧺} ⋯ ⋯
//
\endgl
\xe



\ex \bg
\gla
{}
Pontificis extra octa(vas) transferantur.
Eustachii, quod venit in feriam secundam, ad feriam quintam
post 
//
\glRekonstrukcja
{}
[Pontifi]cis extra octavas transferantur!!!
Eustachii quod venit in feriam secundam, ad[ feriam quintam
post] 
//
\glU
\Upage{5}{3}
cis extra octauā transfferātur Euſtachij quod venit in feriā ſcā aꝺ ⋯ ⋯ ⋯
//
\endgl
\xe



\ex \bg
\gla
{}
Simonis et Judae anteponitur.
Et Leonardi veni(ens) in feriam quintam, ad feriam secundam
//
\glRekonstrukcja
{}
Symonis et jude anteponatur!!!
Et Leonardi veniens in feriam quin[tam, ad feriam secundam]
//
\glU
\Upage{5}{4}
Symonis ⁊ iude anteponatur. ⁊ Leonarꝺi veniēs in feriā quın\Ssgl{⧺} ⋯ ⋯ ⋯
//
\endgl
\xe


\ex \bg
\gla
{}
post easdem octavas postponatur.
Item sancti Clementis veniens in dominicam[dominicum] ibidem teneatur
//
\glRekonstrukcja
{}
post easdem octavas postponatur.
{} Sancti Clementis veniens in domin[icam[dominicum] ibidem teneatur]
//
\glU
\Upage{5}{5}
poſt eaſꝺē octās poſtponatur. {} Sancti Clemētis veniēs in ꝺomin\Ssgl{⧺} ⋯ ⋯
//
\endgl
\xe


\ex \bg
\gla
{}
omeliam do(minicalem) pro nona lectione imponendo. Et quia duae
omeliae dominicales superfluunt, una 
//
\glRekonstrukcja
{}
omeliam dominicalem??? pro nona lectione imponendo. Et quia due
omeliae dom[inicales superfluunt, una]
//
\glU
\Upage{5}{6}
omeliā ꝺominì\Ssgl{⚠}ce ꝓ nona lcōe imponēꝺo. Et quia ꝺue omelie ꝺn̄i\Ssgl{⧺} ⋯ ⋯
//
\endgl
\xe




\ex \bg
\gla
{}
illarum penultima [de][dziura] principe
quarta, et reliqua de panibus sexta feriis se sequentibus
//
\glRekonstrukcja
{}
illarum penultima de principe
quarta et reliqua de panibus. se[xta feriis se sequentibus]
//
\glU
\Upage{5}{7}
illaꝝ penultima ꝺe pꝛincipe quarta ⁊ reliqua vltima ꝺe panibꝰ. ſe\Ssgl{⧺} ⋯ ⋯ ⋯
//
\endgl
\xe



\ex \bg
\gla
{}
sequentibus post festum
sanctae Katherinae suppleantur.
{} Item Andreae apostoli [in dominicam prim]am[dziura]
//
\glRekonstrukcja
{}
[sequen]tibus post festum
sancte Katherinae suppleantur.
� Item andree ap[ostoli [in dominicam prim]am[dziura]]
//
\glU
\Upage{5}{8}
\Ssgl{⧺}tibus poſt feſtū ſancte Katherine ſuppleantur. ⁌ Itē anꝺꝛee ap\Ssgl{⧺} ⋯ ⋯ ⋯
//
\endgl
\xe


\ex \bg
\gla
{}
dominicam prim]am[dziura] Adventus Domini
veniens ad fer[iam][dziura] secundam in crastinum cum toto officio differatur,
//
\glRekonstrukcja
{}
[pri]mam adventus domini
veniens ad feriam secundam in crastinum cum toto [officio differatur,]
//
\glU
\Upage{5}{9}
\Ssgl{⧺}mā aꝺuētus ꝺomini veniens aꝺ feriā ſcam ı\Ssgl{⚠}n craſtinū cū toto ⋯ ⋯
//
\endgl
\xe

% \endinput


\ex \bg
\gla
festum vero per vulgum celebrandum in loco permaneat et omelia vigiliae eiusdem eodem die 
//
\glRekonstrukcja
festum vero per vulgum celebrandum in loco permaneat. et omelia vigi[liae eiusdem eodem die]
//
\glU
\Upage{5}{10}
feſtū vero ꝑ vulgū celebꝛanꝺū ın loco {ꝑ maneat.} ⁊ omelı\Ssgl{⚠}a vigil\Ssgl{⧺} ⋯ ⋯ ⋯
//
\endgl
\xe




\ex \bg
\gla
{}
dominico pro IX. {} lectione imponatur ieiunio
{} sabbato reservato.
{} Item octava Andreae 
//
\glRekonstrukcja
{}
dominico pro {} nona lectione imponatur ieiunio per
sabbato reservato.
� I[tem octava Andreae]
//
\glU
\Upage{5}{11}
ꝺn̄ico ꝓ {} nona lcōe im\Ssgl{⚠}ponatur ieiunio ꝓ ſabbato\Ssgl{⚠} reſeruato. ⁌ I\Ssgl{⧺} ⋯ ⋯
//
\endgl
\xe


\ex \bg
\gla
{}
in dominicam secundam Adventus veniens
ad feriam sextam anticipetur et nihilominus commemoratio
//
\glRekonstrukcja
{}
in dominicam secundam adventus veniens
ad feriam sextam anticipetur.!!! Et [nihilominus commemoratio]
//
\glU
\Upage{5}{12}
in ꝺn̄icā ſcam aꝺn\Ssgl{⚠}entus veniēs aꝺ feriā ſextā anteponatur. Et ⋯ ⋯
//
\endgl
\xe



\ex \bg
\gla
{}
commemoratio de octava
sancti!!! Andreae!!! usque in do(minicam) protendatur.
{} Item festum Conceptionis Immaculatae
//
\glRekonstrukcja
{}
[comm]emoratio de octava
{} {} usque in do(minicam) protendatur.
� Item festum Con[ceptionis Immaculatae]
//
\glU
\Upage{5}{13}
\Ssgl{⧺}memoꝛatio ꝺe octaua {} {} vſ in ꝺn̄ıcā ꝓtenꝺatur. ⁌ Itē feſtū Cō\Ssgl{⧺} ⋯
//
\endgl
\xe



\ex \bg
\gla
{}
Immaculatae Virginis iuxta informationem rubricae anni praeteriti celebretur.
//
\glRekonstrukcja
{}
[Immacu]late virginis iuxta informationem Rubricae anni preteriti celebr[etur.]
//
\glU
\Upage{5}{14}
\Ssgl{⧺}late virginis iuxta informationē Rubꝛice anni preteriti celebꝛ\Ssgl{⧺}
//
\endgl
\xe



\ex \bg
\gla
{}
!!!!(Immaculatae Virginis iuxta informa-
tionem rubricae anni praeteriti celebretur.)!!!
//
\glRekonstrukcja
{}
[?]am octavam pro modus? trium lectionis? tenendo et de aduentu commemo[?]
//
\glU
\Upage{5}{15}
\Ssgl{⚠}am octauam ꝑ moꝺn̄\Ssgl{⚠} triū lcōnū tenenꝺo ⁊ ꝺe aꝺuentu cōmemo\Ssgl{⧺}
//
\endgl
\xe



\ex \bg
\gla
{}
{} Item Luciae sabbato ante dominicam tertiam ve(niens) ibidem
teneatur, sed secundas vesperas tum 
//
\glRekonstrukcja
{}
� Item Lucie Sabbato ante dominicam tertiam veniens ibidem
teneatur {sed secundas vesperas tum}
//
\glU
\Upage{5}{16}
⁌ Itē Lucie Sabbato ante ꝺn̄icā terciā veniēs ibiꝺē teneatur ⋯ ⋯ ⋯ ⋯ 
//
\endgl
\xe

\ex \bg
\gla
{}
ratione sequentis dominicae, tum
ratione antiphonae O sapientia ad Magnificat imponendae
//
\glRekonstrukcja
{}
ratione sequentis dominicae tum
ratione antiphonae. O Sapientia. ad m[agnificat imponendae]
//
\glU
\Upage{5}{17}
ratiōe ſequētis @dominice tū ratiōe anthiphone. O ſapiētia. aꝺ m\Ssgl{⧺} ⋯
//
\endgl
\xe



\ex \bg
\gla
{}
non habebit,
commemorationes tamen Luciae et Conceptionis peragantur.
Item octava Conceptionis
//
\glRekonstrukcja
{}
non habebit,
commemorationes? tamen Lucie et Conceptionis peragantur.
[Item octava Conceptionis]
//
\glU
\Upage{5}{18}
nō habebit. ↄmemoꝛationes t̄n Lucie ⁊ Cōceptiōis ꝑagantur. ⁌ ⋯ ⋯ ⋯
//
\endgl
\xe



\ex \bg
\gla
{}
Conceptionis veniens in feriam secundam post tertiam dominicam Adventus ibidem teneatur more duplicis
//
\glRekonstrukcja
{}
[Con]ceptionis veniens in feriam secundam post tertiam dominicam adventus ib[idem teneatur more duplicis]
//
\glU
\Upage{5}{19}
\Ssgl{⧺}ceptiōis veniēs in feriā ſcdam poſt tertiā ꝺominicā aꝺuentus\Ssgl{⚠} ib\Ssgl{⧺} ⋯ ⋯ ⋯
//
\endgl
\xe



\ex \bg
\gla
{}
festi, {} ita quod
in primis vesperis antiphonae ad psalmos videlicet Nihil est candoris
dicantur
//
\glRekonstrukcja
{}
festi duplicis ita quod
in primis vesperis. antiphonae ad psalmos videlicet nihil es[t candoris
dicantur]
//
\glU
\Upage{5}{20}
feſti ꝺuplicis ita ꝙ in pꝛimis veſꝑis. an̄a aꝺ pſalmos vicꝫ\Ssgl{⚠} nihil eſ\Ssgl{⧺} ⋯ ⋯
//
\endgl
\xe


\ex \bg
\gla
{}
{} {} {} {} {} {} {} {} {} {}
{} {} {} {} {} {} {} {} {} {}
//
\glRekonstrukcja
{}
[?]cum capitulo ut ipso? die. Et respinsorio? ac antiphonam ad magt? ut
//
\glU
\Upage{5}{21}
\Ssgl{⧺}cū capitulo vt ip̄o ꝺie. Et reſpūſoꝛio ac anthiphona aꝺ māgt vt ⋯
//
\endgl
\xe




\ex \bg
\gla
{}
{} {} {} {} {} {} {} {} {} {}
{} {} {} {} {} {} {} {} {} {}
//
\glRekonstrukcja
?
//
\glU
\Upage{5}{22}
\Ssgl{⧺}gilie eiuſꝺē feſti. Cū ↄmemoꝛatiōe. O. anthiphonā diei super ps̈\Ssgl{⚠}. ꝺ\Ssgl{⧺}
//
\endgl
\xe



\ex \bg
\gla
{}
et in crastinum omnia dicantur tam in matutinis
quam in aliis horis, ut ipso die secundae vesperae 
//
\glRekonstrukcja
{}
Et in crastinum omnia dicantur tam in matutinis
quam\Ssgl{⚠} in aliis horis, ut ip[so die secundae vesperae]
//
\glU
\Upage{5}{23}
Et in craſtinū om̄ia ꝺicātur tam in matutīs ̈\Ssgl{⚠} in alijs horis vt ip̄\Ssgl{⧺} ⋯ ⋯ ⋯
//
\endgl
\xe


\ex \bg
\gla
{}
teneantur de Conceptione cum responsorio antiphonam ad Magnificat duplicando et oratione
dicta
//
\glRekonstrukcja
{}
teneantur de conceptione cum responsorio antiphonam ad magnificat dupl[icando et oratione
dicta]
//
\glU
\Upage{5}{24}
teneantur ꝺe ↄceptiōe cū reſpōſoꝛio anthiphonā aꝺ māgt\Ssgl{⚠} ꝺupl\Ssgl{⧺} ⋯ ⋯ ⋯
//
\endgl
\xe




\ex \bg
\gla
{}
dicta et non conclusa responsorium de sancto Lazaro Beatus vir Lazarus
incipiatur et 
//
\glRekonstrukcja
{}
[di]cta et non conclusa responsorium de sancto Lazaro Beatus vir laza[rus
incipiatur et] 
//
\glU
\Upage{5}{25}
\Ssgl{⧺}cta ⁊ nō ↄcluſa reſpōſoꝛium ꝺe ſancto Lazaro. Beatn\Ssgl{⚠}s vir laʒa\Ssgl{⧺} ⋯ ⋯
//
\endgl
\xe



\ex \bg
\gla
{}
vesperae ad finem more solito cum commemoratio(ne) diei
de O sapientia et completorio de Conceptione perficia(n)tur.
//
\glRekonstrukcja
{}
vesperae ad finem more solito cum. commemoratione diei.
{} O? {} {} {} {} perficia(n)tur. et
[???]
//
\glU
\Upage{5}{26}
veſꝑe aꝺ finē moꝛe ſolito. cū ↄmemoꝛatiōe ꝺiei. {} O {} {} {} {} ꝑficiātur. ⁊ ⋯
//
\endgl
\xe



\ex \bg
\gla
\apage{128r}
{} {} In crastino vero de sancto Lazaro per totum teneatur.
Feria quarta 
//
\glRekonstrukcja
\apage{128r}
[?]ptione dicatur.
In crastino vero de sancto Lazaro per totum teneatu[r.
Feria quarta] 
//
\glU
\Upage{5}{27}
ptiōe ꝺicatur. In craſtino vero ꝺe ſancto Laʒaro ꝑ totū teneatu\Ssgl{⧺} ⋯ ⋯
//
\endgl
\xe



\ex \bg
\gla
{}
Quatuor Temporum et quinta et sexta sabbato feriis se
{} sequentibus omeliae dierum
//
\glRekonstrukcja
{}
quatuor Temporum et\Ssgl{⚠}~ quinta et sexta sabbato feriis se
immediante seq[uentibus omeliae dierum]
//
\glU
\Upage{5}{28}
quatuoꝛ tꝑm ac quinta {} ſexta ⁊ ſabbato ferijs ſe immeꝺiante ſeq\Ssgl{⧺} ⋯ ⋯
//
\endgl
\xe



\ex \bg
\gla
{}
dierum cum laudibus propriis teneantur et continuentur.
{} Item Thomae veniens in dominicam quartam
//
\glRekonstrukcja
{}
[die]rum cum laudibus propriis teneantur et continuentur.
� Item Thomae ve[niens in dominicam quartam]
//
\glU
\Upage{5}{29}
\Ssgl{⧺}rū cū lauꝺibus ꝓpꝛijs teneantur ⁊ ↄtinuētur. ⁌ Itē Thomae ve\Ssgl{⧺} ⋯ ⋯ ⋯
//
\endgl
\xe


\ex \bg
\gla
{}
quartam Adventus per vulgum
ibidem celebretur, !!!!
//
\glRekonstrukcja
{}
[quart]am adventus per vulgum
ibidem celebretur, �\Ssgl{⚠} per claram cum toto officio f[?]
//
\glU
\Upage{5}{30}
\Ssgl{⧺}ā aꝺuentus ꝑ vulgū ibiꝺē celebꝛetur. ꝫ\Ssgl{⚠} per claram cū toto officio f\Ssgl{⧺}
//
\endgl
\xe




\ex \bg
\gla
{}
{} {}  Et omelia vigiliae festi eiusdem dominico die pro nona
lectione dicatur et 
//
\glRekonstrukcja
{}
transferratut eiusdem. et omelia vigiliae festi eiusdem dominico die pro no[na
lectione dicatur et]
//
\glU
\Upage{5}{31}
tranſferatut eiuſꝺē. ⁊ omelia vigilie feſti eiuſꝺē ꝺn̄ico ꝺie ꝓ no\Ssgl{⧺} ⋯ ⋯ ⋯
//
\endgl
\xe


\ex \bg
\gla
{}
antiphona Nolite timere die eodem dominico ad Benedictus imponatur pro suffragio 
//
\glRekonstrukcja
{}
antiphona. nolite timere die eodem dominico ad benedictus imp[onatur pro suffragio] 
//
\glU
\Upage{5}{32}
anthiphona. Nolite timere ꝺie eoꝺē ꝺominico aꝺ beneꝺictꝰ īp\Ssgl{⚠}\Ssgl{⧺} ⋯ ⋯
//
\endgl
\xe



\ex \bg
\gla
{}
Beatae Virginis antiphonam Ave Maria
subiungendo. Et eadem feria secunda completis omnibus de festo fiat
comme(moratio) de feria!!!!
//
\glRekonstrukcja
{}
beatae virginis antiphonam. Ave maria
subiungendo. et eadem feria\Ssgl{⚠} s[ecunda completis omnibus de festo fiat
comme(moratio) de feria!!!!]
//
\glU
\Upage{5}{33}
beate virginis anthiphouā. Aue maria ſƀ͡\Ssgl{⚠}\Ssgl{⚠}iungenꝺo. ⁊ eaꝺē ferie ſ\Ssgl{⧺}
//
\endgl
\xe



\ex \bg
\gla
{}
{} {} {} {} {} {} {} {} {} {}
{} {} {} {} {} {} {} {} {} {}
//
\glRekonstrukcja
\Ssgl{⧺}ſti ꝺicta oꝛatiōe ⁊ nō ↄcluſa interponantur lauꝺes ꝺiei ferie ſc\Ssgl{⧺}
//
\glU
\Upage{5}{34}
\Ssgl{⧺}ſti ꝺicta oꝛatiōe ⁊ nō ↄcluſa interponantur lauꝺes ꝺiei ferie ſc\Ssgl{⧺}
//
\endgl
\xe


\ex \bg
\gla
{}
{} {} {} {} {} {} {} {} {} {}
{} {} {} {} {} {} {} {} {} {}
//
\glRekonstrukcja
te incipiantur ⁊ ꝺicantur quin pſalmi feriales laudum cum suis a[?]
//
\glU
\Upage{5}{35}
te incipiantur ⁊ ꝺicantur quin pſalmi feriales laudū cū suis a\Ssgl{⧺}
//
\endgl
\xe

\ex \bg
\gla
{}
{}
{} {} {} {} {} {} {} {} {} {}
{} {} {} {} {} {} {} {} {} {}
//
\glRekonstrukcja
quibus dictis immediate intonetur anthiphona ad benedictus
//
\glU
\Upage{5}{36}
quibus ꝺictis immeꝺiate intonetur anthiphona aꝺ beneꝺictus
//
\endgl
\xe


\ex \bg
\gla
{}
{} {} {} {} {} {} {} {} {} {}
{} {} {} {} {} {} {} {} {} {}
//
\glRekonstrukcja
secunde deferuiencia. Et sic concludautur laudes perversiculum feriale
//
\glU
\Upage{5}{37}
ſce ꝺeferuiēcia. Et \Ssgl{⚠}c concluꝺautur lauꝺes ꝑverculum feriale
//
\endgl
\xe



\ex \bg
\gla
{}
{} Et post eandem quartam dominicam in crastino feria tertia laudes
ultimae terminabuntur.
//
\glRekonstrukcja
{}
ei. Et post candem quartam dominicam in crastino. feria tercia laude
//
\glU
\Upage{5}{38}
\Ssgl{⧺}ei. Et poſt canꝺē quartā ꝺominicā in craſtino. feria terci\Ssgl{⚠}a\Ssgl{⚠} lauꝺe
//
\endgl
\xe




\ex \bg
\gla
terminabuntur.
{} Item responsoria de Clama praedicta feria quarta Quatuor Temporum
incipient et 
//
\glRekonstrukcja
\relax[termina]buntur.
{} Item respon\Ssgl{⚠}soria de Clama praedicta feria quarta Quatu[or]
//
\glU
\Upage{5}{39}
\Ssgl{⧺}buntur. ⁌ Itē Reſposoꝛia ꝺe c�ama p̄ꝺicta feria quarta quatu\Ssgl{⧺}
//
\endgl
\xe



\ex \bg
\gla
perficientur diebus ferialibus ante vigiliam Christi.
{} Item sanctorum Innocentium dominico veniens
//
\glRekonstrukcja
perficientur diebus feriatis\Ssgl{⚠} ante vigiliam Christi.
{} Item sanctorum Innocentium
//
\glU
\Upage{5}{40}
ꝑficientur ꝺiebus feriatis ante vigiliā xp̄i. ⁌ Itē ſanctoꝝ inno\Ssgl{⧺}
//
\endgl
\xe

\ex \bg
\gla
veniens ibidem solenniter
(dictis: Gloria Patri Alleluja ad responsoria longiora et brevia 
//
\glRekonstrukcja
\relax[ve]niens ibidem solenniter
(dictis: Gloria Patri Alleluja ad responsoria [longiora et brevia ]
//
\glU
\Upage{5}{41}
niēs ibiꝺem solenniter: ꝺictis: gloꝛia patri. alꝉia\Ssgl{⚠}. aꝺ reſponſoria ⋯ ⋯ ⋯
//
\endgl
\xe



\ex \bg
\gla
{}
tam in
missa, quam in officio horarum Te Deum laudamus, Gloria in excelsis,
prosa: Ite
//
\glRekonstrukcja
{}
tam in
missa, quam in officio horarum Te Deum laudamus, Gloria in e[xcelsis,
prosa:] 
//
\glU
\Upage{5}{42}
tam in mia ̈ in officio hoꝛaꝝ. Te deum lauꝺamus. gloꝛi\Ssgl{⚠}a in e\Ssgl{⧺} ⋯
//
\endgl
\xe

% now\endinput


\ex \bg
\gla
missa est, his) celebretur, omeliam eiusdem dominicae ad feriam
tertiam proximam differendo[differenda].
//
\glRekonstrukcja
missa est, his: celebretur, omeliam eiusdem dominicae ad feriam
ter[tiam proximam differendo[differenda].]
//
\glU
\Upage{5}{43}
mia eſt his: celebꝛetur. Omeliā eiuſꝺē ꝺominice aꝺ feriam ter\Ssgl{⧺} ⋯ ⋯
//
\endgl
\xe

\ex \bg
\gla
differendo[differenda].
Hoc ordine regatur totus clerus in officio divino per totam dyocesim
Cracoviensem
//
\glRekonstrukcja
\relax[differen]do.
Hoc ordine regatur totus clerus in officio divino per totam [dyocesim
Cracoviensem]
//
\glU
\Upage{5}{44}
ꝺo. Hoc oꝛdine regatur totus clerus in\Ssgl{⚠} officio ꝺiuino per totā ⋯ ⋯
//
\endgl
\xe

\ex \bg
\gla
Cracoviensem nostram anno praesenti ad laudem Dei omnipotentis, qui
est benedictus in saecula saeculorum.
//
\glRekonstrukcja
\relax[Cracovi]ensem nostram anno praesenti ad laudem Dei omnipotentis, qui
est ben[edictus in saecula saeculorum.]
//
\glU
\Upage{5}{45}
\Ssgl{⧺}enſem noſtrā anno pꝛeſenti aꝺ laudė\Ssgl{⚠} ꝺei om̄ipotētis qui eſt ben\Ssgl{⧺} ⋯ ⋯ ⋯
//
\endgl
\xe

\ex \bg
\gla
{} {} {} {} {} {} {} {} {} {}
{} {} {} {} {} {} {} {} {} {}
//
\glRekonstrukcja
Sapiente Salomonis Stef pro do iuj
//
\glU
\Upage{6}{1}
Sapiente Salomonis Stef pꝛo ꝺo iiy\Ssgl{⚠}
//
\endgl
\xe

\ex \bg
\gla
{} {} {} {} {} {} {} {} {} {}
{} {} {} {} {} {} {} {} {} {}
//
\glRekonstrukcja
Si bona Iob E giꝺium ij
//
\glU
\Upage{6}{2}
Si bona Iob E giꝺium ij
//
\endgl
\xe

\ex \bg
\gla
{} {} {} {} {} {} {} {} {} {}
{} {} {} {} {} {} {} {} {} {}
//
\glRekonstrukcja
�eto Thobie � j
//
\glU
\Upage{6}{3}
{} Thobie \Ssgl{⧺}nicen j
//
\endgl
\xe

\endinput




%%% Local Variables:
%%% mode: latex
%%% ispell-local-dictionary: "polish"
%%% coding: utf-8-unix
%%% TeX-PDF-mode: t
%%% TeX-engine: xetex
%%% TeX-master: "Rubricella_parallel"
%%% default-input-method: "OldPolish"
%%% End:



\printbibliography

%\printendnotes

%\printindex

\end{document}



%%% Local Variables:
%%% mode: latex
%%% ispell-local-dictionary: "polish"
%%% coding: utf-8-unix
%%% TeX-PDF-mode: t
%%% TeX-engine: xetex
%%% TeX-master: t
%%% End:
